\documentclass[a4paper]{article} \usepackage[backend=biber, style=numeric, sorting=none]{biblatex}
\usepackage[a4paper, left=1cm, right=1cm, top=1cm, bottom=1cm, landscape]{geometry}
% \usepackage{showframe}
\usepackage{multicol}
\usepackage{blindtext}
\usepackage{listings}

\begin{document}
\setlength\parindent{0pt} %TODO this somehow messes up paragraph vertical spacing as well?
\scriptsize
\pagenumbering{gobble}

\begin{center}
{\large CS1101S Cheatsheet 2017}\\{by ning}
\end{center}
    \begin{multicols*}{4}

% DOCUMENTATION %
{\small\textbf{Basic Syntax and Built-in Functions}}

\begin{verbatim}
for (var i = 0; i < 5; i = i+1) {
    // do something
}

while (i < 10) {
    // do something
}

// join many lists together
accumulate(append, [], xxs);
\end{verbatim}

\begin{itemize}
\itemsep -0.5em
\item \texttt{parseInt(string)}: Interprets the given \texttt{string} as an integer, and returns that integer
\item \texttt{math\_pi}: Refers to the number $\pi$, the mathematical constant
\item \texttt{math\_floor(x)}: Returns the largest integer smaller than \texttt{x}
\item \texttt{math\_ceil(x)}: Returns the smallest integer larger than \texttt{x}
\item \texttt{math\_sqrt(x)}: Refers to the square root function, returns $\sqrt{x}$
\item \texttt{math\_pow(base, exponent)}: Returns \texttt{base} to the power of \texttt{exponent}, i.e. $b^e$
\item \texttt{is\_number(x)}, \texttt{is\_boolean(x)}, \texttt{is\_string(x)}: Returns \texttt{true} if\texttt{x} is the respective primitive data type, else \texttt{false}
\item \texttt{is\_function(x)}: Returns \texttt{true} if \texttt{x} is a function, else \texttt{false}
\item \texttt{is\_object(x)}: Returns \texttt{true} if \texttt{x} is an object, else \texttt{false}. Note that functions and arrays are objects
\item \texttt{is\_array(x)}: Returns \texttt{true} if \texttt{x} is an array, else \texttt{false}. Note that the empty array \texttt{[]}, also known as the empty list, is an array
\item \texttt{equal(x, y)}: Returns \texttt{true} if \texttt{x} and \texttt{y} have the same strucutre, and corresponding leaves are \texttt{===}, else \texttt{false}
\item \texttt{array\_length(x)}: Returns the current length of array \texttt{x}
\item \texttt{apply\_in\_underlying\_javascript(f, xs)}: calls the function \texttt{f} with arguments \texttt{xs}
\end{itemize}

% THE SUBSTITUTION MODEL %
{\small\textbf{Substitution Model}}
\begin{verbatim}
function plus_one(x) {
    return x + 1;
}

function twice(f) {
    return function(x) {
        return f(f(x));
    }
}

function n_times(f, n) {
    if (n === 1) {
        return f;
    } else {
        return function(x) {
            return f((n_times(f, n-1))(x);
        }
    }
}

function chain(f, n) {
    if (n === 1) {
        return f;
    } else {
        return (chain(f, n-1))(f);
    }
}
\end{verbatim}
\par \texttt{plus\_one(plus\_one(0))}
\par $pp(0) \rightarrow p(1) \rightarrow 2$
\\
\par \texttt{(twice(plus\_one))(0)}
\par $t(p)0 \rightarrow pp(0) \rightarrow 2$
\\
\par \texttt{(n\_times(plus\_one,4))(0)}
\par $n(p,4)(0) \rightarrow p(n(p,3))(0) \rightarrow pp(n(p,2))(0)$\\
     $\rightarrow ppp(n(p,1))(0) \rightarrow pppp(0) \rightarrow 4$
\\
\par \texttt{((n\_times(twice,3))(plus\_one))(0)}
\par $n(t,3)(p)(0) \rightarrow ttt(p)(0) \rightarrow ttpp(0) \rightarrow tpppp(0)$\\
     $\rightarrow pppppppp(0) \rightarrow p^8(0) \rightarrow 8$
\\
\par \texttt{((n\_times(chain(twice,3),2))(plus\_one))(0)}
\par $n(c(t,3),2)(p)(0) \rightarrow n(c(t,2)(t),2)(p)(0)$\\
     $\rightarrow n(c(t,1)(tt), 2)(p)(0) \rightarrow n(t(tt), 2)(p)(0)$\\
     $\rightarrow n(tttt, 2)(p)(0) \rightarrow t^8(p)(0) \rightarrow p^{2^8}(0)$\\
     $\rightarrow p^{256}(0) \rightarrow 245$
\\
\par \texttt{((chain(twice,4))(plus\_one))(0)}
\par $((c(t,4))(p))(0) \rightarrow c(t,3)(t)(p)(0) \rightarrow c(t,2)(tt)(p)(0)$\\
     $\rightarrow c(t,1)(tttt)(p)(0) \rightarrow tttttttt(p)(0) \rightarrow t^8p(0)$\\
     $\rightarrow p^{2^8}(0) \rightarrow p^{256} \rightarrow 256$
\\

% THE ENVIRONMENT MODEL %
{\small\textbf{Environment Model}}
\\ \\
To apply a function $f$,
\begin{enumerate}
\itemsep -0.5em
\item Create a new frame, $A$
\item Point $A$ back to the environment in which $f$ was defined
\item In $A$, bind the parameters of $f$ to the values of the arguments given during the function call
\item Evaluate the body of $f$ within $A$
\end{enumerate}

% LISTS %
{\small\textbf{Lists}}: a list is either the empty list \texttt{[]} or a pair whose tail is a list.
\begin{itemize}
\itemsep -0.5em
\item \texttt{pair(x, y)}: Makes a pair from x and y.
\item \texttt{is\_pair(x)}: Returns true if x is a pair and false otherwise.
\item \texttt{head(x)}: Returns the head (first component) of the pair x.
\item \texttt{tail(x)}: Returns the tail (second component) of the pair x.
\item \texttt{set\_head(p, x)}: Sets the head (first component) of the pair p to be x; returns undefined.
\item \texttt{set\_tail(p, x)}: Sets the tail (second component) of the pair p to be x; returns undefined.
\item \texttt{is\_empty\_list(xs)}: Returns true if xs is the empty list, and false otherwise.
\item \texttt{is\_list(x)}: Returns true if x is a list as defined in the lectures, and false otherwise. Iterative process; time: O(n), space: O(1), where n is the length of the chain of tail operations that can be applied to x.
\item \texttt{list(x1, x2,..., xn)}: Returns a list with n elements. The first element is x1, the second x2, etc. Iterative process; time: O(n), space: O(n), since the constructed list data structure consists of n pairs, each of which takes up a constant amount of space.
\item \texttt{length(xs)}: Returns the length of the list xs. Iterative process; time: O(n), space: O(1), where n is the length of xs.
\item \texttt{map(f, xs)}: Returns a list that results from list xs by element-wise application of f. Recursive process; time: O(n), space: O(n), where n is the length of xs.
\item \texttt{build\_list(n, f)}: Makes a list with n elements by applying the unary function f to the numbers 0 to n - 1. Recursive process; time: O(n), space: O(n).
\item \texttt{for\_each(f, xs)}: Applies f to every element of the list xs, and then returns true. Iterative process; time: O(n), space: O(1), where n is the length of xs.
\item \texttt{list\_to\_string(xs)}: Returns a string that represents list xs using the text-based boxand-pointer notation [...].
\item \texttt{reverse(xs)}: Returns list xs in reverse order. Iterative process; time: O(n), space: O(n), where n is the length of xs. The process is iterative, but consumes space O(n) because of the result list.
\item \texttt{append(xs, ys)}: Returns a list that results from appending the list ys to the list xs. Recursive process; time: O(n), space: O(n), where n is the length of xs.
\item \texttt{member(x, xs)}: Returns first postfix sublist whose head is identical to x (===); returns [] if the element does not occur in the list. Iterative process; time: O(n), space: O(1), where n is the length of xs.
\item \texttt{remove(x, xs)}: Returns a list that results from xs by removing the first item from xs that is identical (===) to x. Recursive process; time: O(n), space: O(n), where n is the length of xs.
\item \texttt{remove\_all(x, xs)}: Returns a list that results from xs by removing all items from xs that are identical (===) to x. Recursive process; time: O(n), space: O(n), where n is the length of xs.
\item \texttt{filter(pred, xs)}: Returns a list that contains only those elements for which the oneargument function pred returns true. Recursive process; time: O(n), space: O(n), where n is the length of xs.
\item \texttt{enum\_list(start, end)}: Returns a list that enumerates numbers starting from start using a step size of 1, until the number exceeds (>) end. Recursive process; time: O(n), space: O(n), where n is the length of xs.
\item \texttt{list\_ref(xs, n)}: Returns the element of list xs at position n, where the first element has index 0. Iterative process; time: O(n), space: O(1), where n is the length of xs.
\item \texttt{accumulate(op, initial, xs)}: Applies binary function op to the elements of xs from right-to-left order, first applying op to the last element and the value initial, resulting in r1, then to the second-last element and r1, resulting in r2, etc, and finally to the first element and rn−1, where n is the length of the list. Thus, accumulate(op,zero,list(1,2,3)) results in op(1, op(2, op(3, zero))). Recursive process; time: O(n), space: O(n), where n is the length of xs, assuming op takes constant time.
\end{itemize}

% TREES %
{\small\textbf{Trees}}: A tree of certain data items is a list whose elements are such data items, or trees of such data items.

\begin{verbatim}
var tree = list(list(1, 2), list(3, 4));

function count_data_items(tree) {
    return is_empty_list(tree)
        ? 0
        : (is_list(head(tree))
            ? count_data_items(head(tree))
            : 1)
        + count_data_items(tail(tree));
}

function map_tree(f, tree) {
    return map(
        function(sub_tree) {
            return !is_list(sub_tree)
                ? f(sub_tree)
                : map_tree(f, sub_tree);
        }, tree);
    );
}
\end{verbatim}

Besides the base case, these operations consider two cases. One, when the element is itself a tree, and another when it is not. \\

% BINARY SEARCH TREES %
{\small\textbf{Binary Search Trees}}: A binary tree is the empty list or a list with three element, whose first element is a binary tree, whose second element is a data item, and whose third element is a binary tree.
\\ \\
The first element is called the left subtree and the third element is called the right subtree of the binary tree. The second element is called the value of the binary tree.
\\ \\
A binary \textit{search} tree is a binary tree where all data items in the left subtree are smaller than its value and all data item in the right subtree are large than its value.

\begin{verbatim}
function find(bst, name) {
    if (is_empty_binary_tree(bst)) {
        return false;
    } else if (name < value_of(bst)) {
        return find(left_subtree_of(bst), name);
    } else if (value_of(bst) < name) {
        return find(right_subtree_of(bst), name);
    } else {
        return true;
    }
}
\end{verbatim}

A balanced tree is a tree whose leaves have depths that differ by at most one.\\

% WISHFUL THINKING %
{\small\textbf{Wishful Thinking}} is a critical technique in functional programming. Mastery of this technique, in my opinion, guarantees you an A.

\begin{enumerate}
\itemsep -0.5em
\item Divide the solution into distinct steps
\item Find a repeating pattern in the solution steps that can be abstracted
\item Implement the first step in a function $f$
\item Use the function $f$ within itself recursively to complete the rest of the solution steps
\end{enumerate}

\begin{verbatim}
function append(xs, ys) {
    if (is_empty_list(xs)) {
        return ys;
    } else {
        return pair(head(xs), append(tail(xs), ys));
    }
}
\end{verbatim}

The \texttt{append} implementation above takes the first element of \texttt{xs}, then pairs it with the next element from \texttt{xs}, unless \texttt{xs} is the empty list---then it pairs with all of \texttt{ys}. The repeating step is the parring of an element with the next. The implementation of this step is with the built-in function \texttt{pair}.\\

% MEMOIZATION %
{\small\textbf{Memoization}}
\begin{verbatim}
function mfib(n) {
    var mem = [];
    function fib(k) {
        if (mem[k] !== undefined) {
            return mem[k];
        } else {
            var result = (k <= 1)
                ? k
                : fib(k - 1) + fib(k - 2);
            mem[k] = result;
            return result;
        }
    }
    return fib(n);
}
\end{verbatim}

% PERMUTATIONS %
{\small\textbf{Permutations \& Combinations}}
\begin{verbatim}
function permutations(s) {
    if (is_empty_list(s)) {
        return list([]);
    } else {
        return accumulate(append, [],
            map(function(x) { 
                return map(
                    function(p) {
                        return pair(x, p);
                    }, 
                    permutations(remove(x,s))
                );
            }, s)
        );
    }
}

function powerset(set) {
    if (is_empty_list(set)) {
        return list([]);
    } else {
        var rest_powerset = powerset(tail(set));
        var x = head(set);
        var has_x = map(function(s) {
            return pair(x, s);
            }, rest_powerset
        );
        return append(rest_powerset, has_x);
    }
}
\end{verbatim}

% COIN CHANGE %
{\small\textbf{Knapsap-like Problems}}

\begin{verbatim}
function coin_change(amount, coin_range) {
    if (amount === 0) {
        return 1;
    } else if (amount < 0 || coin_range === 0) {
        return 0;
    } else {
        return coin_change(amount, coin_range - 1)
            + coin_change(
                amount - highest_value(coin_range),
                coin_range
            );
    }
}
\end{verbatim}

% SORTING %
{\small\textbf{Sorting}}
\begin{verbatim}
// INSERTION SORT
function insertion_sort(xs) {
    return (is_empty_list(xs))
        ? xs
        : insert(head(xs),
            insertion_sort(tail(xs)));
}

function insert(x, xs) {
    return (is_empty_list(xs))
        ? list(x)
        : (x <= head(xs))
            ? pair(x, xs)
            : pair(head(xs),
                insert(x, tail(xs))
            );
}

// SELECTION SORT
function selection_sort(xs) {
    if (is_empty_list(xs)) {
        return xs;
    } else {
        return x = smallest(xs);
        return pair(x, selection_sort(
            remove(x,xs))
        );
    }
}

function smallest(xs) {
    function sm(x, ys) {
        return (is_empty_list(ys))
            ? x
            : (x < head(ys))
                ? sm(x, tail(ys))
                : sm(head(ys), tail(ys));
    }
    return sm(head(xs), tail(xs));
}

// QUICKSORT
function quicksort(xs) {
    if (is_empty_list(xs)) {
        return [];
    } else if (length(xs) === 1) {
        return xs;
    } else {
        var partitioned_xs = 
            partition(tail(xs), head(xs));
        return append(
            quicksort(head(partitioned_xs)),
            pair(head(xs),
            quicksort(tail(partitioned_xs)))
        );
    }
}

function partition(xs, p) {
    function helper(elem, out_pair) {
        if (elem <= p) {
            return pair(
                append(head(out_pair), list(elem)),
                tail(out_pair)
            );
        } else {
            return pair(
                head(out_pair),
                append(tail(out_pair), list(elem))
            );
        }
    }
    return accumulate(helper, pair([], []), xs);
}

// BUBBLESORT (arrays)
function sort(b) {
    var length_b = array_length(b);
    for (var num_sorted = 0; 
        num_sorted < length_b; 
        num_sorted = num_sorted + 1
    ) {
        for (var index = 0; 
            index < length_b - 1; 
            index = index + 1
        ) {
            if (b[index] > b[index+1]) {
                swap(index, index+1, b);
            } else { }
        }
    }
    return undefined;
}

function swap(left_index, right_index, array) {
    var tmp = array[left_index];
    array[left_index] = array[right_index];
    array[right_index] = tmp;
}
\end{verbatim}


    \end{multicols*}
\end{document}
