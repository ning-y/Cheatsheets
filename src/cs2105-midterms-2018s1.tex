\documentclass[a4paper]{article} \usepackage[backend=biber, style=numeric, sorting=none]{biblatex}
\usepackage[a4paper, left=0.8cm, right=0.8cm, top=0.8cm, bottom=0.8cm, landscape]{geometry}
% \usepackage{showframe}
\usepackage{multicol}
\usepackage{blindtext}
\usepackage{listings}
\usepackage{graphicx}
\usepackage{enumitem}
\graphicspath{{images/cs2105/}}

\begin{document}
\setlength\parindent{0pt} %TODO this somehow messes up paragraph vertical spacing as well?
\scriptsize
% \tiny
\pagenumbering{gobble}

\begin{center}
{\large CS2105 Cheatsheet 18/19 S1 Midterms}\\{by vig}
\end{center}
    \begin{multicols*}{4}

% APPLICATION LAYER %
{\small\textbf{Application Layer}}
\textbf{Processes}
\begin{itemize}[leftmargin=*]
\itemsep -0.5em
\item Applications send \texttt{messages} to each other using \texttt{sockets}
\item Application processes can only control:
  \begin{itemize}[leftmargin=*]
  \item transport protocol used
  \item minor transport-layer parameters
  \end{itemize}
\item To send a message to another application process we need:
  \begin{itemize}[leftmargin=*]
  \item IP Address of the host
  \item Destination port number
  \end{itemize}
\end{itemize}

\textbf{Transport services in general}
\begin{itemize}[leftmargin=*]
\item If there is Reliable Data Transfer (TCP) $\Rightarrow$ No loss
\item If there is high throughput $\Rightarrow$ Large amounts of data can be transferred at a time
\item If there is timing/latency guarantee $\Rightarrow$ interactive applications feel realistic
\item Protocol can specify security e.g encryption, checksum verification, end-point authentication
\end{itemize}

\textbf{TCP}
\begin{itemize}[leftmargin=*]
\item A TCP Handshake must be formed before two-way connection between client and host
\item Reliable Data Transfer: data is received in proper order without erroneous, duplicate or missing byes
\item Has a flow-control and congestion-control mechanism
\end{itemize}

\textbf{UDP}
\begin{itemize}[leftmargin=*]
\item Lightweight and connectionless
\item Unreliable data transfer service: no guarantee of reaching in the correct order, or even reaching in the first place
\item No congestion-control mechanism
\end{itemize}

\textbf{HTTP}
\begin{itemize}[leftmargin=*]
\item Stateless protocol -- cookies hold state (e.g preserve shopping cart)
\item Common statuses 200 OK, 301 Moved Permanently, 400 Bad Request, 403 Forbidden,404 Not Found, 500 Internal Server Error
\item A Web Cache keeps copies of recently requested objects in this storage, and makes TCP handshakes with server to get the object (if it's not cached)
\item Non-persistence (1.0): One handshake for establishing connection, then \texttt{one file} is requested, followed by another handshake to close connection
\item Persistence (1.1): One handshake for establishing connection, then files are requested sequentially. Connection is left open by server
\item Pipelining (1.1): New requests made before previous ones are resolved, but order of objects is preserved (by TCP)
\item Multiplexing (2): Responses can come back in any order, or even partially
\end{itemize}

\textbf{DNS}
\begin{itemize}[leftmargin=*]
\item DNS (Domain Name Server) holds Resource Records (RR)
  \begin{itemize}[leftmargin=*]
  \item (Name, Value, Type, TTL)
  \item \texttt{Type=A} $\Rightarrow$ Name: hostname, Value: IP
  \item \texttt{Type=NS} $\Rightarrow$ Name: domain, Value: IP of authoritative DNS
  \item \texttt{Type=CNAME} $\Rightarrow$ Name: alias hostname, Value: canonical hostname
  \item \texttt{Type=MX} $\Rightarrow$ Name: alias hostname, Value: canonical name of mail server
  \end{itemize}
\item Root servers direct queries to TLD servers, only ~13 root servers in the world
\item TLD (Top-level Domain) responsible for \texttt{.com, .org, .net, .sg...}
\item Authoritative server keeps hostname-IP mappings of organization's named hosts
\item Local DNS acts as a proxy, caches previously retrieved mappings to cache. This allows for faster lookup. Owned by ISPs
\item All DNS servers can implement caching, so root servers are often not called upon
\end{itemize}

    \end{multicols*}
\end{document}
