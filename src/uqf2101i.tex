\documentclass{article}

\usepackage[
    a4paper, left=1cm, right=1cm, top=1cm, bottom=1cm, landscape
]{geometry}
\usepackage{multicol}
\usepackage{mhchem}
\usepackage{color, soul}  % enable highlighting
                          % note: does not work with \ce{}

\newcommand{\headingsmall}[1]{{\small\textbf{#1}}}

\begin{document}

\scriptsize                % Small fonts
\pagenumbering{gobble}     % No page numbers
\setlength\parindent{0pt}  % No indents at start of paragraphs

% TITLE %
\begin{center}
{\large UQF2101I Cheatsheet}\\{for test 1, by ning}
\end{center}

% BODY %
\begin{multicols*}{4}

%% Learning Objectives %%
\headingsmall{Learning Objectives}
\begin{itemize} \itemsep -0.5em
    % QR Framework
    \item Know how to approach a problem quantitatively
    \item Use QR tools to critically think about that problem
    \item To solve (or at least make a good effort at solving!) the problem
    % Air Quality
    \item Recognise the criteria air pollutants and their adverse effects
    \item Know how air quality is quantified and reported
    % Beach Water Quality
    \item Recognise key indicators of beach water quality
    \item Know how beach water quality is quantified
    \item Qualitatively discuss our observations/questions of interest to us
    \item Be able to formulate hypothesis to `answer' our questions
    \item Be able to deduce predictions for testing our hypotheses
\end{itemize}

%% Quantitative Reasoning %%
\headingsmall{Quantitative Reasoning} is \textit{
    the way in which we can use numbers to provide evidence for our 
    arguments.
} It has 10 elements,
\begin{enumerate} \itemsep -0.5em
    \item Scientific method
    \item Hypothesis
    \item Sampling
    \item Data
    \item Operationalize
    \item Variable
    \item Modelling
    \item Analysis
    \item Deduction
    \item Critical Thinking
\end{enumerate}
This module examines `pressing environmental issues that
\textit{continue to have a significant impact on human health}. QR
provides\dots
\begin{itemize} \itemsep -0.5em
    \item more clarity, better structured, objective, and useful 
        analysis of issues, 
    \item so that decision-making and arguments can be guided and
        supported by quantitative proof and rigourous analysis
    \item proof with numbers that a problem exists, and the magnitude
        of that problem
\end{itemize}

%% Scientific Method %%
The \headingsmall{scientific method} invloves four generally 
sequential steps:
\begin{enumerate} \itemsep -0.5em
    \item \textbf{Observation/Question}---start off by noticing a phenomenon
        that poses a problem/question
    \item \textbf{Hypothesis}---form a \textit{suggested explanation} of the
        observation above; made on the basis of limited evidence or
        survey of existing literature; \hl{must be falsifiable}
    \item \textbf{Prediction}---then, deduce a specific prediction
        \hl{based on the hypothesis}
    \item \textbf{Empirical Test}---everything else\dots
\end{enumerate}

%% Operationalise %%
In order to test the hypothesis, we need to translate theoretical
concepts to actual measures through \headingsmall{operationalisation}.
By measuring these to obtain values, we can \hl{quantify} those concepts. \\

%% Variables %%
A \headingsmall{variable} is something whose value can vary
\begin{itemize} \itemsep -0.5em
    \item A \hl{random variable} is a variable that can take on one or more
        values, each associated with a certain \hl{probability}
    \item Variables are either \hl{dependent or independent}
    \item Dependent variables corresponds to the outcomes we want to explain, 
        they are determined by other variables
    \item Independent variables vary independently of other variables, they
        determine (fully or partially) the dependent variable
\end{itemize}

%% Air Quality %%
\headingsmall{Air Quality} problems all begin with the emission of
pollutants. Pollutants are substances in the air that \hl{change the
original composition of the air around us, and has an adverse effect on
human health, or the environment}. 
\begin{itemize} \itemsep -0.5em
    \item Excessive levels of pollutants can cause adverse health 
        effects. Singapore tracks 6 criteria health pollutants: 
        \ce{CO, NO_2, O_3, SO_2, PM_{2.5}, PM_{10}}. 
    \item Sources of these air pollutants can be categorised as 
        \hl{mobile or stationary}, and \hl{primary or secondary}. 
        Primary sources directly emit pollutants into the air, 
        secondary sources are the formation of pollutants through
        physical or chemical processes in the air.
    \item Air quality control measures aim to limit airborne 
        concentrations to \hl{a level where no adverse health effects 
        are expected}.
    \item Air control measures can be \hl{source reduction, or
        air-scrubbing (indirect)}
\end{itemize}

%% Carbon Monoxide %%
\headingsmall{Carbon Monoxide, \ce{CO}}
\begin{itemize} \itemsep -0.5em
    \item \hl{Non-reactive species} in urban air
    \item Does not react rapidly with surfaces
    \item Low solubility in air
    \item \hl{Atmospheric concentrations vary directly with emissions}
    \item Main fate is oxidation to \ce{CO_2}
    \item Binds irreversibly with haemoglobin
    \item \hl{Impairs visual perception, work capacity, manual dexterity, 
        learning ability, and performance of complex tasks}
    \item \hl{Life-threatening at high levels (about 600 ppm)}
    \item Elderly, young, and those with pre-existing cardiovascular diseases most
        sensitive
    \item \hl{From incomplete combustion of hydrocarbon-based fuels}
    \item From (predominantly) vehicle exhaust
    \item From cigarette smoke (indoor pollutant)
    \item \hl{Control by catalytic converters to treat exhaust}
    \item Control by inspection programmes (smog check)
    \item Control by using reformulated, oxygenated fuels
\end{itemize}

%% Nitrogen Dioxide %%
\headingsmall{Nitrogen Dioxide, \ce{NO_2}}
\begin{itemize} \itemsep -0.5em
    \item Most nitrous oxide missions are in the form \ce{NO}, with remainder
        \ce{NO_2}. However, since \ce{NO} quickly oxidises to \ce{NO_2} in
        the air, nitrous oxide concentrations are reported and measured as
        \ce{NO_2} concentrations
    \item \hl{Respiratory irritant} that lowers resistance to respiratory
        infections when inhaled
    \item Long-term exposure to elevated levels may cause increased
        incidence of acute respiratory diseases in children
    \item Contributes to \hl{formation of ozone}
    \item Causes visibility problems (reddish-brown colour in smog)
    \item Forms acid deposition (rain): \ce{NO_2 + OH -> HNO_3}
    \item From \hl{oxidation of NO} in the air, and \ce{NO} is a combustion
        by-product from industries, power plants, and motor vehicles
    \item Control by combustion manipulation---using the minimum mix of
        atmospheric air (rich in \ce{N_2}) for complete combustion
    \item Control by \hl{catalytic converters} reducing \ce{NO_X} back to \ce{N_2}
\end{itemize}

%% Ozone %%
\headingsmall{Ozone, \ce{O_3}}
\begin{itemize} \itemsep -0.5em
    \item Refers to low altitude ozone
    \item Strong oxidant that cause \hl{chemical damage} to lung tissue
        and makes lungs more sensitive to other pollutants/irritants
    \item Can cause chest pain, coughing and nausea
    \item Long-term exposure may lead to \hl{permanent structural damage of lungs}
    \item Damages plantlife (crops, trees) as well
    \item From \hl{secondary sources only: photochemical reactions} in the atmosphere,
        with \ce{NO_X} and VOC (volatile organic compounds) as reactants
    \item VOC from various sources, such as vehicle exhaust, \hl{fuel/solvent
        evaporation}, industrial processes, plant emissions
    \item Control by reducing emissions of reactants \ce{NO_X} and VOCs
    \item Control by \hl{controlling VOC} through product substitution/
        reformulation, sorption by activated carbon, better vapour containment, or
        catalytic oxidation
\end{itemize}

%% Sulfur Dioxide %%
\headingsmall{Sulfur Dioxide, \ce{SO_2}}
\begin{itemize} \itemsep -0.5em
    \item Highly water soluble, likely to be absorbed by moist surfaces of upper 
        respiratory tract
    \item But, can travel deeper into lungs when entrained in particles
    \item Affects breathing, causes respiratory illnesses, alters defense
        mechanisms of lungs, aggrevates pre-existing symptons
    \item Dominant precursor for acid deposition
    \item From \hl{combustion of sulfur-containing fuels} (for electricity/industry)
    \item From combustion of high sulfur fuel by vehicles and off-road equipment
        (minor source)
    \item Control by \hl{reducing sulfur content} in fuels
    \item Treating flue gas to remove \ce{SO_2}
\end{itemize}

%% Particulate Matter %%
\headingsmall{Particulate Matter, \ce{PM_{10} \& PM_{2.5}}}
\begin{itemize} \itemsep -0.5em
    \item Number following `PM' is the `size' of the particle in \ce{\mu m}
    \item Particle \hl{behaviour dependent on particle size}
    \item Course particles $> 2.5 \mu \mathrm{m}$ are controlled by mass and
        inertia, have lifetime of days to weeks before settling or impacting; 
        therefore, effect mostly on areas near its source
    \item Ultrafine particles $< 0.1 \mu \mathrm{m}$ readily diffuse and coalesce
        with other particles; have lifetime of a few days to weeks
    \item Intermediate particles $0.1 - 2.5 \mu \mathrm{m}$ \hl{too small to settle,
        too light to impact, too large to diffuse; have lifetime on the order of
        weeks; can be transported across large distances before depositing}
    \item Adverse effects on breathing, aggrevate pre-existing respiratory and
        cardio vascular diseases, particularly asthma
    \item Impairs immune system, mechanical damage to lung tissue
    \item Major contributor against atmospheric visibility
    \item From primary sources, for coarse: road dust, construction, industrial,
        and agricultural emissions
    \item From primary sources, for fine: industrial emissions, diesel soot from
        vehicles, wood burning
    \item From \hl{secondary sources, for fine}: oxidation of hydrocarbons into
        organics, formation of aerosol salts from \ce{SO_2} and \ce{NO_X}
    \item Control by filtration mechanisms such as electrostatic precipitation,
        mechanical filtration
\end{itemize}

%% Pollutant Standards Index %%
\headingsmall{Pollutant Standards Index, PSI}
\begin{center}\begin{tabular}{l|c|c|c}
    Air Quality  & PSI       & 24-hr \ce{PM_{10}} & 24-hr \ce{SO_2} \\
    \hline
    Good         & $0-50$    & $0-50$             & $0-80$          \\
    Moderate     & $51-100$  & $51-150$           & $81-365$        \\
    Unhealthy    & $101-200$ & $151-350$          & $366-800$       \\
    V. Unhealthy & $201-300$ & $351-420$          & $801-1600$      \\
    Hazardous    & $301-400$ & $421-500$          & $1601-2100$     \\
    Hazardous    & $401-500$ & $501-600$          & $2101-2620$     \\
\end{tabular}\end{center}
\begin{center}\begin{tabular}{l|c|c|c}
    Air Quality  & PSI       & 8-hr \ce{CO} & 8-hr \ce{O_3} \\
    \hline
    Good         & $0-50$    & $0.0-5.0$    & $0-118$       \\
    Moderate     & $51-100$  & $5.1-10.0$   & $119-157$     \\
    Unhealthy    & $101-200$ & $10.1-17.0$  & $158-235$     \\
    V. Unhealthy & $201-300$ & $17.1-34.0$  & $236-785$     \\
    Hazardous    & $301-400$ & $34.1-46.0$  & $786-980$     \\
    Hazardous    & $401-500$ & $46.1-57.5$  & $981-1180$    \\
\end{tabular}\end{center}
\begin{center}\begin{tabular}{l|c|c|c}
    Air Quality  & PSI       & 1-hr \ce{NO_2} & 24-hr \ce{PM_{2.5}} \\
    \hline
    Good         & $0-50$    & --             & $0-12$              \\
    Moderate     & $51-100$  & --             & $13-55$             \\
    Unhealthy    & $101-200$ & $1130$         & $56-150$            \\
    V. Unhealthy & $201-300$ & $1131-2260$    & $151-250$           \\
    Hazardous    & $301-400$ & $2261-3000$    & $251-350$           \\
    Hazardous    & $401-500$ & $3001-3750$    & $251-500$           \\
\end{tabular}\end{center}
All measures are in $\mu\mathrm{g/m^3}$, except for \ce{CO} which is in
$\mathrm{mg/m^3}$.
$$ I_i = 
    \frac{I_{i, j+1} - I_{i, j}}{X_{i, j+1} - X_{i, j}} 
    \cdot (X_i - X_{i,j})
    + I_{i, j} $$
\begin{itemize} \itemsep -0.5em
    \item \hl{Relates health effects} with varying levels of pollution
    \item Provides \hl{health advisories} to the public on the necessary precautions
        when pollution levels go into unsafe levels
    \item Average pollutant concentration is converted into a sub-index, highest
        sub-index for a region is taken as its PSI
    \item Monitored through \textit{Telemetric Air Quality Monitoring Network}, 22
        stations islandwide place according to population density
\end{itemize}

\end{multicols*}
\end{document}
