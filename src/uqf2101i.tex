\documentclass{article}

\usepackage[
    a4paper, left=1cm, right=1cm, top=1cm, bottom=1cm, landscape
]{geometry}
\usepackage{multicol}
\usepackage{mhchem}
\usepackage{IEEEtrantools}
\usepackage{color, soul}  % enable highlighting
                          % note: does not work with \ce{}

\newcommand{\headingsmall}[1]{{\small\textbf{#1}}}

\begin{document}

\scriptsize                % Small fonts
\pagenumbering{gobble}     % No page numbers
\setlength\parindent{0pt}  % No indents at start of paragraphs

% TITLE %
\begin{center}
{\large UQF2101I Cheatsheet}\\{for test 1, by ning}
\end{center}

% BODY %
\begin{multicols*}{4}

%% Learning Objectives %%
\headingsmall{Learning Objectives}
\begin{itemize} \itemsep -0.5em
    % QR Framework
    \item Approach a problem quantitatively
    \item Critically think about problem w/ QR tools
    \item To solve (or make a good effort) the problem
    % Air Quality
    \item Criteria air pollutants and their adverse effects
    \item Know how air quality is quantified and reported
    % Beach Water Quality
    \item Recognise key indicators of beach water quality
    \item Know how beach water quality is quantified
    % Scientific Method
    \item Qualitatively discuss observations/questions
    \item Able to formulate hypothesis
    \item Able to deduce predictions for hypotheses
    % Empirical Tests
    \item Describe the different methods used to collect data for testing 
        our predictions
    \item Explain the advantages/disadvantages of the various methods of
        data collection
    \item Identify the various factors that can affect the validity of outcomes
        in our study
    % Probability
    \item Relate probability to quantitative reasoning
    \item Able to perform probability calculations
    \item Conditional probability \& independence
    % Basic Statistics
    \item Preliminary analysis of data
    \item Descriptive statistics
    % Graphical methods
    \item Graphical methods
\end{itemize}

%% Quantitative Reasoning %%
\headingsmall{Quantitative Reasoning} is \textit{
    the way in which we can use numbers to provide evidence for our 
    arguments.
} It has 10 elements,
\begin{enumerate} \itemsep -0.5em
    \item Scientific method
    \item Hypothesis
    \item Sampling
    \item Data
    \item Operationalize
    \item Variable
    \item Modelling
    \item Analysis
    \item Deduction
    \item Critical Thinking
\end{enumerate}
This module examines `pressing environmental issues that
\textit{continue to have a significant impact on human health}. QR
provides\dots
\begin{itemize} \itemsep -0.5em
    \item more clarity, better structured, objective, and useful 
        analysis of issues, 
    \item so that decision-making and arguments can be guided and
        supported by quantitative proof and rigorous analysis
    \item proof with numbers that a problem exists, and the magnitude
        of that problem
\end{itemize}

%% Scientific Method %%
The \headingsmall{scientific method} involves four generally 
sequential steps:
\begin{enumerate} \itemsep -0.5em
    \item \textbf{Observation/Question}---start off by noticing a phenomenon
        that poses a problem/question
    \item \textbf{Hypothesis}---form a \textit{suggested explanation} of the
        observation above; made on the basis of limited evidence or
        survey of existing literature; \hl{must be falsifiable}
    \item \textbf{Prediction}---then, deduce a specific prediction
        \hl{based on the hypothesis}
    \item \textbf{Empirical Test}---using everything else
\end{enumerate}

%% Operationalise %%
In order to test the hypothesis, we need to translate theoretical
concepts to actual measures through \headingsmall{operationalisation}.
By measuring these to obtain values, we can \hl{quantify} those concepts. \\

%% Variables %%
A \headingsmall{variable} is something whose value can vary
\begin{itemize} \itemsep -0.5em
    \item A \hl{random variable} is a variable that can take on one or more
        values, each associated with a certain \hl{probability}
    \item Variables are either \hl{dependent or independent}
    \item Dependent variables corresponds to the outcomes we want to explain, 
        they are determined by other variables
    \item Independent variables vary independently of other variables, they
        determine (fully or partially) the dependent variable
\end{itemize}

%% Air Quality %%
\headingsmall{Air Quality} problems all begin with the emission of
pollutants. Pollutants are substances in the air that \hl{change the
original composition of the air around us, and has an adverse effect on
human health, or the environment}. 
\begin{itemize} \itemsep -0.5em
    \item Excessive levels of pollutants can cause adverse health 
        effects. Singapore tracks 6 criteria health pollutants: 
        \ce{CO, NO_2, O_3, SO_2, PM_{2.5}, PM_{10}}. 
    \item Sources of these air pollutants can be categorised as 
        \hl{mobile or stationary}, and \hl{primary or secondary}. 
        Primary sources directly emit pollutants into the air, 
        secondary sources are the formation of pollutants through
        physical or chemical processes in the air.
    \item Air quality control measures aim to limit airborne 
        concentrations to \hl{a level where no adverse health effects 
        are expected}.
    \item Air control measures can be \hl{source reduction, or
        air-scrubbing (indirect)}
\end{itemize}

%% Carbon Monoxide %%
\headingsmall{Carbon Monoxide, \ce{CO}}
\begin{itemize} \itemsep -0.5em
    \item \hl{Non-reactive species} in urban air
    \item Does not react rapidly with surfaces
    \item Low solubility in air
    \item Atmospheric concentrations vary directly with emissions
    \item Main fate is oxidation to \ce{CO_2}
    \item Binds irreversibly with haemoglobin
    \item Impairs visual perception, work capacity, manual dexterity, 
        learning ability, and performance of complex tasks
    \item \hl{Life-threatening at high levels (about 600 ppm)}
    \item Elderly, young, and those with pre-existing cardiovascular diseases most
        sensitive
    \item \hl{From incomplete combustion of hydrocarbon-based fuels}
    \item From (predominantly) vehicle exhaust
    \item From cigarette smoke (indoor pollutant)
    \item \hl{Control by catalytic converters to treat exhaust}
    \item Control by inspection programmes (smog check)
    \item Control by using reformulated, oxygenated fuels
\end{itemize}

%% Nitrogen Dioxide %%
\headingsmall{Nitrogen Dioxide, \ce{NO_2}}
\begin{itemize} \itemsep -0.5em
    \item Most nitrous oxide missions are in the form \ce{NO}, with remainder
        \ce{NO_2}. However, since \ce{NO} quickly oxidises to \ce{NO_2} in
        the air, nitrous oxide concentrations are reported and measured as
        \ce{NO_2} concentrations
    \item \hl{Respiratory irritant} that lowers resistance to respiratory
        infections when inhaled
    \item Long-term exposure to elevated levels may cause increased
        incidence of acute respiratory diseases in children
    \item Contributes to \hl{formation of ozone}
    \item Causes visibility problems (reddish-brown colour in smog)
    \item Forms acid deposition (rain): \ce{NO_2 + OH -> HNO_3}
    \item From \hl{oxidation of NO} in the air, and \ce{NO} is a combustion
        by-product from industries, power plants, and motor vehicles
    \item Control by combustion manipulation---using the minimum mix of
        atmospheric air (rich in \ce{N_2}) for complete combustion
    \item Control by \hl{catalytic converters} reducing \ce{NO_X} back to \ce{N_2}
\end{itemize}

%% Ozone %%
\headingsmall{Ozone, \ce{O_3}}
\begin{itemize} \itemsep -0.5em
    \item Refers to low altitude ozone
    \item Strong oxidant that cause \hl{chemical damage} to lung tissue
        and makes lungs more sensitive to other pollutants/irritants
    \item Can cause chest pain, coughing and nausea
    \item Long-term exposure may lead to \hl{permanent structural damage of lungs}
    \item Damages flora (crops, trees) as well
    \item From \hl{secondary sources only: photochemical reactions} in the atmosphere,
        with \ce{NO_X} and VOC (volatile organic compounds) as reactants
    \item VOC from various sources, such as vehicle exhaust, \hl{fuel/solvent
        evaporation}, industrial processes, plant emissions
    \item Control by reducing emissions of reactants \ce{NO_X} and VOCs
    \item Control by \hl{controlling VOC} through product substitution/
        reformulation, sorption by activated carbon, better vapour containment, or
        catalytic oxidation
\end{itemize}

%% Sulfur Dioxide %%
\headingsmall{Sulfur Dioxide, \ce{SO_2}}
\begin{itemize} \itemsep -0.5em
    \item Highly water soluble, likely to be absorbed by moist surfaces of upper 
        respiratory tract
    \item But, can travel deeper into lungs when entrained in particles
    \item Affects breathing, causes respiratory illnesses, alters defence
        mechanisms of lungs, aggravates pre-existing symptoms
    \item Dominant precursor for acid deposition
    \item From \hl{combustion of sulfur-containing fuels} (for electricity/industry)
    \item From combustion of high sulfur fuel by vehicles and off-road equipment
        (minor source)
    \item Control by \hl{reducing sulfur content} in fuels
    \item Treating flue gas to remove \ce{SO_2}
\end{itemize}

%% Particulate Matter %%
\headingsmall{Particulate Matter, \ce{PM_{10} \& PM_{2.5}}}
\begin{itemize} \itemsep -0.5em
    \item Number following `PM' is the `size' of the particle in \ce{\mu m}
    \item Particle \hl{behaviour dependent on particle size}
    \item Course particles $> 2.5 \mu \mathrm{m}$ are controlled by mass and
        inertia, have lifetime of days to weeks before settling or impacting; 
        therefore, effect mostly on areas near its source
    \item Ultrafine particles $< 0.1 \mu \mathrm{m}$ readily diffuse and coalesce
        with other particles; have lifetime of a few days to weeks
    \item Intermediate particles $0.1 - 2.5 \mu \mathrm{m}$ \hl{too small to settle,
        too light to impact, too large to diffuse; have lifetime on the order of
        weeks; can be transported across large distances before depositing}
    \item Adverse effects on breathing, aggravates pre-existing respiratory and
        cardiovascular diseases, particularly asthma
    \item Impairs immune system, mechanical damage to lung tissue
    \item Major contributor against atmospheric visibility
    \item From primary sources, for coarse: road dust, construction, industrial,
        and agricultural emissions
    \item From primary sources, for fine: industrial emissions, diesel soot from
        vehicles, wood burning
    \item From \hl{secondary sources, for fine}: oxidation of hydrocarbons into
        organics, formation of aerosol salts from \ce{SO_2} and \ce{NO_X}
    \item Control by filtration mechanisms such as electrostatic precipitation,
        mechanical filtration
\end{itemize}

%% Pollutant Standards Index %%
\headingsmall{Pollutant Standards Index, PSI}
\begin{center}\begin{tabular}{l|c|c|c}
    Air Q.  & PSI       & 24-hr \ce{PM_{10}} & 24-hr \ce{SO_2} \\
    \hline
    Good    & $0-50$    & $0-50$             & $0-80$          \\
    Mod.    & $51-100$  & $51-150$           & $81-365$        \\
    Unh.    & $101-200$ & $151-350$          & $366-800$       \\
    V. Unh. & $201-300$ & $351-420$          & $801-1600$      \\
    Hazard. & $301-400$ & $421-500$          & $1601-2100$     \\
    Hazard. & $401-500$ & $501-600$          & $2101-2620$     \\
\end{tabular}\end{center}
\begin{center}\begin{tabular}{l|c|c|c}
    Air Q.  & PSI       & 8-hr \ce{CO} & 8-hr \ce{O_3} \\
    \hline
    Good    & $0-50$    & $0.0-5.0$    & $0-118$       \\
    Mod.    & $51-100$  & $5.1-10.0$   & $119-157$     \\
    Unh.    & $101-200$ & $10.1-17.0$  & $158-235$     \\
    V. Unh. & $201-300$ & $17.1-34.0$  & $236-785$     \\
    Hazard. & $301-400$ & $34.1-46.0$  & $786-980$     \\
    Hazard. & $401-500$ & $46.1-57.5$  & $981-1180$    \\
\end{tabular}\end{center}
\begin{center}\begin{tabular}{l|c|c|c}
    Air Q.  & PSI       & 1-hr \ce{NO_2} & 24-hr \ce{PM_{2.5}} \\
    \hline
    Good    & $0-50$    & --             & $0-12$              \\
    Mod.    & $51-100$  & --             & $13-55$             \\
    Unh.    & $101-200$ & $1130$         & $56-150$            \\
    V. Unh. & $201-300$ & $1131-2260$    & $151-250$           \\
    Hazard. & $301-400$ & $2261-3000$    & $251-350$           \\
    Hazard. & $401-500$ & $3001-3750$    & $251-500$           \\
\end{tabular}\end{center}
All measures are in $\mu\mathrm{g/m^3}$, except for \ce{CO} which is in
$\mathrm{mg/m^3}$.
$$ I_i = 
    \frac{I_{i, j+1} - I_{i, j}}{X_{i, j+1} - X_{i, j}} 
    \cdot (X_i - X_{i,j})
    + I_{i, j} $$
\begin{itemize} \itemsep -0.5em
    \item \hl{Relates health effects} with varying levels of pollution
    \item Provides \hl{health advisories} to the public on the necessary precautions
        when pollution levels go into unsafe levels
    \item Average pollutant concentration is converted into a sub-index, highest
        sub-index for a region is taken as its PSI
    \item Monitored through \textit{Telemetric Air Quality Monitoring Network}, 22
        stations islandwide place according to population density
\end{itemize}

%% Beach Water Quality %%
\headingsmall{Beach Water Quality}
\begin{itemize} \itemsep -0.5em
    \item Most common pollutant is \hl{faecal pollution}
    \item From sources such as sewage leaks, decomposition, wildlife, humans, 
        marine vessels dumping waste, industrial wastes, surface run-off
    \item Can cause diarrhea, vomiting, respiratory infections, infections of
        the eyes/ears/nose/skin; or gastroenteritis
    \item Levels of faecal pollution can be measured by indicators of 
        coliform bacterial or faecal \textit{streptococci}
    \item Total coliforms---can occur in human faeces but also naturally e.g. 
        in soil or submerged wood; therefore not recommended
    \item Faecal coliforms---a subset of total coliforms more specific to
        human faeces, but contains genus \textit{Klebsiella} which may also
        occur naturally e.g. textiles, pulp, and paper mill wastes
    \item \hl{\textit{E. Coli}}---a species of faecal coliform specific to human
        and mammal faeces; most specific measure of coliform bacteria; currently
        recommended by USEPA
    \item Faecal \textit{streptococci}---generally occur in digestive systems
        of mammals
    \item \hl{\textit{Enterococci}}---a genus more human-specific than faecal
        \textit{streptococci}, previously categorised under the 
        genus \textit{streptococcus}; currently recommended by USEPA as a
        indicator for salt water, since these more more \hl{able to survive in
        salt water environments}
\end{itemize}

%% Singapore's Recreational Water Guidelines %%
\headingsmall{Singapore Recreational Water Guidelines}
\begin{itemize} \itemsep -0.5em
    \item 95\% of the time, \textit{Entorococcus} counts for the last three years
        must be less than or equal \hl{200 counts per 100ml}; these are rated 
        `Good' or better
    \item Also takes into account susceptibility of the location to faecal influence
    \item Only `Good' or better beaches are considered suitable for \hl{primary 
        contact activities}
    \item Beaches are sampled weekly
\end{itemize}

%% Hong Kong's Beach Grading System
\headingsmall{Hong Kong Beach Grading System}
\begin{center}\begin{tabular}{c|c|c|c}
Grd & BWQ   & n/100ml                & Ill/1000ppl \\
\hline
1   & Good  & $\leq$ 24              & $-$         \\
2   & Fair  & $24-180$               & $\leq 10$   \\
3   & Poor  & $181-610$              & $11 - 15$   \\
4   & V. Pr & $<610$ or last $>1600$ & $> 15$
\end{tabular}\end{center}
\begin{itemize} \itemsep -0.5em
    \item Based on geometric mean of 5 most recent samples, or if last reading
        very high
    \item Public advised not to swim in Grade 4 beaches until quality improves
    \item During bathing season (March to October), beaches monitored at least
        three times a month, interval between sampling 3 to 14 days
    \item During non-bathing season (November to February), beaches which
        remain open still monitored at least three times a month, but other
        beaches monitored once per month
\end{itemize}

%% Empirical Tests %%
\headingsmall{Empirical Tests} are conducted on predictions made (through the
scientific method), using data\dots
\begin{itemize} \itemsep -0.5em
    \item Data---numerical values of physical quantities; \hl{collected through}
         retrospective studies, observational studies, or designed experiments
    \item \hl{Retrospective studies}---looks at limited historical records
    \item Hard to determine a `sole impact', because lack of control over variables
    \item Independent variable may not have varied historically
    \item Form of data may not be ideal (want daily, have yearly)
    \item May not get most relevant data (want diesel vehicles, have all vehicles)
    \item Overall, difficult to obtain solid and reliable results; hard, if not impossible
        to conclusively prove causal relationships
    \item \hl{Observational studies}---observe population of interest with minimal 
        disturbance
    \item Usually carried out over a short period of time
    \item Allows measurements of some variables normally unavailable to retrospective studies
    \item \hl{Designed experiments}---able to make deliberate changes to independent
        variables, in order to observe change in dependent variable
    \item Good control over what to measure, and how
    \item \hl{Time range to collect} influenced by practicality, periodic trends 
        (e.g. seasonal), relevance to current conditions (e.g. data from 1900s vs 2000s)
    \item \hl{Type of data chosen} influenced by relevance (e.g. rainfall near 
        \textit{E. Coli} sampling sites); can include control measurements
    \item Generally, three factors leading to \hl{invalidation} of data
    \item \hl{Measurement}---such as counting errors, contamination
    \item \hl{Sampling}---such as proximity of rainfall station, sample representative 
        \textit{E. Coli} samples of the whole beach, using outdated data
    \item \hl{Operationalise}---such as choosing 12h, 24h, 48h, or 96h rainfall
    \item Followed by modelling, analysis, and critical thinking (e.g. examine assumptions
        made)
\end{itemize}

%% Probability %%
\headingsmall{Probability}
\begin{itemize} \itemsep -0.5em
    \item Relative frequency of an outcome settles down to one value over the long run
    \item The one value is then the probability of that outcome
\end{itemize}

%% Basic Statistics %%
\headingsmall{Basic Statistics}
\begin{itemize} \itemsep -0.5em
    \item Presents data economically
    \item Set of numbers to summarise data and its characteristics
    \item Three types: \hl{Measures of central tendency, dispersion
        and asymmetry}
    \item Central tendency---a value that many systems tend to cluster
        around; can be taken as being \hl{representative of data collected}
    \item Sample arithmetic mean---strongly \hl{influenced by outliers}
    \item Median---also the $50^{\mathrm{th}}$ percentile; less sensitive
         to outliers (although $95^{\mathrm{th}}$ percentile is 
        \textit{more} sensitive to outliers
    \item Geometric mean---less sensitive to outliers, always $\leq$ the
        arithmetic mean; used in averaging values that represent 
        rate of change
    \item Mode---most frequently occurring values, can be more than one
    \item A. Mean $=$ Median $\implies$ symmetrical distribution; other
        direction not always true
    \item Range---extremely susceptible to outliers $\implies$ mainly
        used for small $n$
    \item Interquartile range---difference of the $1^\mathrm{st}$ and
        $3^\mathrm{rd}$ percentile; more resistant to outliers
    \item Variance ($s^2/\sigma^2$) and standard deviation ($s/\sigma$)
        ---sensitive to extreme values
    \item Data can be considered to be \hl{dispersed if \textit{coefficient of 
        variation} $>$ 1}
\end{itemize}

%% Graphical Methods %%
\headingsmall{Graphical Methods}
\begin{itemize} \itemsep -0.5em
    \item Dotplot---good for small pool of data, for spotting unusual data points,
        bad for spotting dispersion pattern
    \item Boxplot---good for highlighting distribution features, for comparing
        data measuring related or similar characteristics; box shows $1^\mathrm{st}$,
     $2^\mathrm{nd}$ and $3^\mathrm{rd}$ quartiles, whiskers to first data point
        within 1.5 IQR of box borders, rest are outliers
    \item Histogram---good for observing shape of distribution, central tendency and scatter,
        for guessing type of probability distribution, bad for small bin size;
        \hl{recommended number of bins is $\sqrt{n}$}, traditionally bottom 
        inclusive and top exclusive (but reversed for excel)
    \item Data points can be considered to be \hl{outliers if outside 
        $[\bar x-3s, \bar x+3s]$}
    \item Skew---a measure of symmetry; positive indicates right skew, tail to the right,
        negative indicates left skew, tail to the left; zero is symmetrical
    \item Time series---a plot of data against time, since time may be an important
        factor in variability; can \hl{compare trends in mean and dispersion} through time
\end{itemize}

%% Other Statistical Terms %%
\headingsmall{Other Statistical Terms}
\begin{itemize} \itemsep -0.5em
    \item Variability---can be higher-end, lower-end
    \item Sampling---choosing subset of population to characterise 
        properties of population
    \item Sample space---all possible outcomes
    \item Event---subset of the sample space
    \item Outliers---data that do not seem to conform to distribution
        of the rest of the data; due to errors in data collection 
        \textit{(above)}
\end{itemize}

%% Formulas %%
\headingsmall{Formulas} \\[5pt]
% Percentile
{
% Reduce before and after between equations
\setlength{\abovedisplayskip}{3pt}
\setlength{\belowdisplayskip}{3pt}
\textbf{Percentile Calculations}
    \begin{IEEEeqnarray*}{rCl}
    \mathrm{Percentile\ Rank,\ r} &=&
        1 + p\ (n-1) \\
    \mathrm{p^{th}\ percentile} &=&
        (1 - (r\ \mathrm{mod}\ 1))\ X_{\mathrm{int}(r)}\ + \\
        && \mathrm{int}(r) \cdot X_{\mathrm{int}(r)+1}
    \end{IEEEeqnarray*}
\textbf{Basic Probability}
    \begin{IEEEeqnarray*}{rCCCl}
    E_1\ & \mathrm{OR}   &\ E_2 &=& E_1 \cup E_2 \\
    E_1\ & \mathrm{AND}  &\ E_2 &=& E_1 \cap E_2 \\
         & \mathrm{NOT}  &\ E   &=& E'
    \end{IEEEeqnarray*}
    \begin{IEEEeqnarray*}{rCl}
    \mathrm{P}(B|A) &=& 
        \frac{\mathrm{P}(A\cap B)}{\mathrm{P}(A)}
    \end{IEEEeqnarray*}
\textbf{Mutual Exclusitivity}
    \begin{IEEEeqnarray*}{rrCl}
    \mathrm{if}  & E_1 \cap E_2 &=& 0 \\
    \mathrm{then}&\  \mathrm{P}(E_1 \cup E_2) &=&
        \mathrm{P}(E_1) + \mathrm{P}(E_2) \\
    \mathrm{and} & \mathrm{P}(E_1 \cap E_2) &=& 0
    \end{IEEEeqnarray*} 
\textbf{Independence}
    \begin{IEEEeqnarray*}{rCl}
        &&     \mathrm{E_1\ and\ E_2\ are\ independent} \\
        &\iff& \mathrm{P}(E_1|E_2) = \mathrm{P}(E_1) \\
        &\iff& \mathrm{P}(E_2|E_1) = \mathrm{P}(E_2) \\
        &\iff& \mathrm{P}(E_1 \cap E_2) =
                \mathrm{P}(E_1)\mathrm{P}(E_2)
    \end{IEEEeqnarray*}
\textbf{Basic Statistics}
    \begin{IEEEeqnarray*}{rCl}
        \bar x_a &=& \frac{1}{n}(x_1 + x_2 + \cdots + x_n) \\
                 &=& \frac{1}{n}\sum^n_{i=1}x_i \\
        \bar x_g &=& \sqrt[n]{x_1 \cdot x_2 \cdot \cdots x_n} \\
                 &=& \left ( \prod^n_{i=1} x_i \right )^{\frac{1}{n}}
    \end{IEEEeqnarray*}
    \begin{IEEEeqnarray*}{rCl}
        s^2 &=& \frac{1}{(n-1)}\sum^n_{i=1}(x_i - \bar x)^2 \\
              &=& \frac{1}{(n-1)} 
        \left [
            \sum^n_{i=1} x_i^2 
                - \frac{\left ( \sum^n_{i=1} x_i \right )^2}{n}
        \right ]
    \end{IEEEeqnarray*}
    \begin{IEEEeqnarray*}{C}
        \mathrm{CV} = \frac{s}{\bar x} \times 100\%\ 
        \mathrm{or}\ \frac{\sigma}{\mu} \times 100\%
    \end{IEEEeqnarray*}
}

\end{multicols*}
\end{document}
